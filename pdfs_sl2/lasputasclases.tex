
\documentclass{article}


\usepackage{lipsum}
\usepackage{lmodern}
\usepackage[T1]{fontenc}
\usepackage[utf8]{inputenc}
\usepackage{mathtools}
\usepackage{amssymb}
\usepackage{amsmath}
\usepackage{graphicx}
\usepackage{sectsty} % Para cambiar el tamanio de los titulos
\usepackage[lmargin=1in, rmargin=0.75in]{geometry}
\usepackage{xcolor}
%\usepackage{hyperref}
\usepackage[colorlinks = true,
linkcolor = black,
urlcolor  = blue,
citecolor = blue,
anchorcolor = blue]{hyperref}

\sectionfont{\LARGE}
\newcommand{\rpm}{\raisebox{.2ex}{$\scriptstyle\pm$}} % rpm is a smaller pm
\newcommand\tab[1][0.6cm]{\hspace*{#1}}
\newcommand\nl{\newline\tab}
\newcommand{\bigslant}[2]{{\raisebox{.2em}{$#1$}\left/\raisebox{-.2em}{$#2$}\right.}}
\newcommand{\suchthat}{\;\ifnum\currentgrouptype=16 \middle\fi|\;}

\renewcommand\labelenumi{\textbf{\arabic{enumi}.}}
\renewcommand\labelenumii{\textbf{\theenumi.\arabic{enumii}.}}

\renewcommand{\theenumii}{\textbf{.\arabic{enumii}.}}
%\renewcommand{\labelenumi}{\alph{enumi}}


\renewcommand\thesubsection{\arabic{subsection}.} 
\renewcommand\thesection{\arabic{section}} %Para que las secciones no se nombren como chapter.section (0.1, 0.2 etc)
\renewcommand\labelitemi{$\bullet$ }

\setcounter{secnumdepth}{3}

\linespread{1.2}

\title{No sé muy bien ni qué quiero hacer.}
\date{}
\begin{document}
	\maketitle
	%\begin{center}
	Voy a estudiar $ \bigslant{SL_2(\mathbb{Z})}{\mathbb{S}}$, donde  
	$$\mathbb{S} = 
	\langle
	\begin{pmatrix}
	1 & 2 \\
	0 & 1 
	\end{pmatrix},
	\begin{pmatrix}
	1 & 0 \\
	2 & 1 
	\end{pmatrix}
	\rangle = \left\{ 
	\begin{pmatrix}
	a & b \\
	c & d 
	\end{pmatrix}
	\suchthat
	\begin{split}
	a \equiv d \equiv 1 &\mod 4 \\
	b \equiv c \equiv 0 &\mod 2 
	\end{split}
	\right\}
	$$
	\nl
	Para ello, voy a hacer un estudio exhaustivo y nada agradable de la clase de equivalencia donde acabaría cierta matriz $	\mathcal{M}$ en función de sus coeficientes módulo 2 y módulo 4.\nl
	A lo mejor no la hago taaaan exhaustiva, y sólo hago unos pocos casos.
	\nl 
	
	\subsection{Las clases de equivalencia}
	Sea $ \mathcal{M} = 
	\begin{pmatrix}
	a & b \\
	c & d 
	\end{pmatrix} $
	\begin{enumerate}
		\item Si $b \equiv c \equiv 0 \mod 2$:
		\begin{align*}
		\phi(\mathcal{M}) = 
		\begin{pmatrix}
			x & 0 \\
			0 & t
		\end{pmatrix}  & \implies
		\phi(\overline{\mathcal{M}}) = 
		\phi(\mathcal{M}\mathbb{S}) =
		\begin{pmatrix}
			x & 0 \\
			0 & t 
		\end{pmatrix}\\ 
		& 	\implies \varphi(\overline{\mathcal{M}}) = 	
		\left\{
		\begin{pmatrix}
		x & 0 \\
		0 & t 
		\end{pmatrix},
		\begin{pmatrix}
		x & 0 \\
		2 & t 
		\end{pmatrix},
		\begin{pmatrix}
		x & 2 \\
		0 & t 
		\end{pmatrix},
		\begin{pmatrix}
		x & 2 \\
		2 & t 
		\end{pmatrix}
		\right\}	
		\\ & \implies \overline{\mathcal{M}} \subseteq 
		\varphi^{-1}\left(	\left\{
		\begin{pmatrix}
		x & y \\
		z & t 
		\end{pmatrix}
		\suchthat
		(y,z) \in \{0,2\}
		\right\}	
		\right) 
		\end{align*}
		Lo chuliguay es que $\subseteq $ es un $=$, ya que ninguna clase $\overline{\mathcal{N}} \neq \overline{\mathcal{M}}$ tiene un representante $\mathcal{N} \in \overline{\mathcal{N}}$ tal que  $\varphi(\mathcal{N}) \in \varphi(\overline{\mathcal{M}})$. No voy a repetir ese argumento, y a partir de ahora pondré siempre los $=$.\newline \newline	El caso es que 
		$
		\overline{\mathcal{M}} =
		\varphi^{-1}\left( 
		\left\{
		\begin{pmatrix}
		x & y \\
		z & t 
		\end{pmatrix}
		\suchthat
		(y,z) \in \{0,2\}
		\right\}	
		\right)$ y que como 
		$ \begin{pmatrix}
		x & y \\
		z & t 
		\end{pmatrix} 
		\in SL_2(\mathbb{Z}/ 4\mathbb{Z}) $, sabemos que $(x, t) = (1,1)$ o $(x, t) = (3,3)$.\nl 
		
		\begin{enumerate}
			\item Si $(x, t) = (1,1)$, tenemos una clase de equivalencia 
			$$
			\overline{\mathcal{M}_{1}} = 
			\varphi^{-1}\left( 
			\left\{
			\begin{pmatrix}
			1 &0 \\
			0 & 1 
			\end{pmatrix},
			\begin{pmatrix}
			1 & 0 \\
			2 & 1 
			\end{pmatrix},
			\begin{pmatrix}
			1 & 2 \\
			0 & 1 
			\end{pmatrix},
			\begin{pmatrix}
			1 & 2 \\
			2 & 1 
			\end{pmatrix},
			\right\}	
			\right) $$
			
			\item  Si $(x, t) = (3,3)$, tenemos una clase de equivalencia 
			$$
			\overline{\mathcal{M}_{2}} = 
			\varphi^{-1}\left( 
			\left\{
			\begin{pmatrix}
			3 &0 \\
			0 & 3 
			\end{pmatrix},
			\begin{pmatrix}
			3 & 0 \\
			2 & 3 
			\end{pmatrix},
			\begin{pmatrix}
			3 & 2 \\
			0 & 3 
			\end{pmatrix},
			\begin{pmatrix}
			3 & 2 \\
			2 & 3 
			\end{pmatrix},
			\right\}	
			\right) $$
			
		\end{enumerate}
		
		\item Si $b \equiv0 \mod 2,  c \equiv 1 \mod 2$:
		\begin{align*}
		\phi(\mathcal{M}) = 
		\begin{pmatrix}
		x & 0 \\
		1 & t 
		\end{pmatrix}  & \implies
		\phi(\overline{\mathcal{M}}) = 
		\phi(\mathcal{M}\mathbb{S}) = \left\{
		\begin{pmatrix}
		x & 0 \\
		1 & t 
		\end{pmatrix},
		\begin{pmatrix}
		x & 0 \\
		1 & t+2 
		\end{pmatrix},
		\right\} \\ & \implies
		\varphi(\overline{\mathcal{M}}) =
		\left\{
		\begin{pmatrix}
		x & y \\
		z & n
		\end{pmatrix} \suchthat 	
		\begin{aligned}
		n &\in \{t, t+2\} \\
		y &\in \{0, 2\} \\
		z &\in \{1, 3\} 
		\end{aligned}					
		\right\} 
		\end{align*}
		
		En este caso, tenemos que 
		$
		\overline{\mathcal{M}} =
		\varphi^{-1}\left( 
		\left\{
		\begin{pmatrix}
		x & y \\
		z & n
		\end{pmatrix}
		\suchthat
		\begin{aligned}
		n &\in \{t, t+2\} \\
		y &\in \{0, 2\} \\
		z &\in \{1, 3\} 
		\end{aligned}
		\right\}	
		\right)$ y que como 
		$ \begin{pmatrix}
		x & y \\
		z & n 
		\end{pmatrix} 
		\in SL_2(\mathbb{Z}/ 4\mathbb{Z}) $, sabemos que $(x, t) = (1,1), (1,3), (3,1) $ o $ (3,3)$.\newline Según esto:
		\begin{enumerate}
			\item Si $(x,t) = (1,1)$ o $(x,t) = (1,3)$, tenemos una clase de equivalencia 
			$$
			\overline{\mathcal{M}_{3}} = 
			\varphi^{-1}\left( 
			\left\{
			\begin{pmatrix}
			1 & 0 \\
			1 & 1 
			\end{pmatrix},
			\begin{pmatrix}
			1 & 0 \\
			3 & 1 
			\end{pmatrix},
			\begin{pmatrix}
			1 & 2 \\
			1 & 3 
			\end{pmatrix},
			\begin{pmatrix}
			1 & 2 \\
			3 & 3 
			\end{pmatrix},
			\right\}	
			\right) $$
			\item Si $(x,t) = (3,1)$ o $(x,t) = (3,3)$, tenemos una clase de equivalencia 
			$$
			\overline{\mathcal{M}_{4}} = 
			\varphi^{-1}\left( 
			\left\{
			\begin{pmatrix}
			3 & 2 \\
			1 & 1 
			\end{pmatrix},
			\begin{pmatrix}
			3 & 2 \\
			3 & 1 
			\end{pmatrix},
			\begin{pmatrix}
			3 & 0 \\
			1 & 3 
			\end{pmatrix},
			\begin{pmatrix}
			3 & 0 \\
			3 & 3 
			\end{pmatrix},
			\right\}	
			\right) $$
		\end{enumerate}
	
	
		\item Si $b \equiv 1 \mod 2,  c \equiv 0 \mod 2$:
		\begin{align*}
		\phi(\mathcal{M}) = 
		\begin{pmatrix}
		x & 0 \\
		1 & t 
		\end{pmatrix}  & \implies
		\phi(\overline{\mathcal{M}}) = 
		\phi(\mathcal{M}\mathbb{S}) = \left\{
		\begin{pmatrix}
		x & 0 \\
		1 & t 
		\end{pmatrix},
		\begin{pmatrix}
		x+2 & 1 \\
		0 & t 
		\end{pmatrix},
		\right\} \\ & \implies
		\varphi(\overline{\mathcal{M}}) =
		\left\{
		\begin{pmatrix}
		m & y \\
		z & t
		\end{pmatrix} \suchthat 	
		\begin{aligned}
		m &\in \{x, x+2\} \\
		y &\in \{1, 3\} \\
		z &\in \{0, 2\} 
		\end{aligned}					
		\right\} 
		\end{align*}
		
		En este caso, tenemos que 
		$
		\overline{\mathcal{M}} =
		\varphi^{-1}\left( 
		\left\{
		\begin{pmatrix}
		m & y \\
		z & t
		\end{pmatrix}
		\suchthat
		\begin{aligned}
		m &\in \{x, x+2\} \\
		y &\in \{1, 3\} \\
		z &\in \{0, 2\} 
		\end{aligned}
		\right\}	
		\right)$ y que como 
		$ \begin{pmatrix}
		m & y \\
		z & t 
		\end{pmatrix} 
		\in SL_2(\mathbb{Z}/ 4\mathbb{Z}) $, sabemos que $(x, t) = (1,1), (1,3), (3,1) $ o $ (3,3)$.\newline Según esto:
		\begin{enumerate}
			\item Si $(x,t) = (1,1)$ o $(x,t) = (3,1)$, tenemos una clase de equivalencia 
			$$
			\overline{\mathcal{M}_{5}} = 
			\varphi^{-1}\left( 
			\left\{
			\begin{pmatrix}
			1 & 1 \\
			0 & 1 
			\end{pmatrix},
			\begin{pmatrix}
			1 & 3 \\
			0 & 1 
			\end{pmatrix},
			\begin{pmatrix}
			3 & 1 \\
			2 & 1 
			\end{pmatrix},
			\begin{pmatrix}
			3 & 3 \\
			2 & 1 
			\end{pmatrix},
			\right\}	
			\right) $$
			\item Si $(x,t) = (1,3)$ o $(x,t) = (3,3)$, tenemos una clase de equivalencia 
			$$
			\overline{\mathcal{M}_{6}} = 
			\varphi^{-1}\left( 
			\left\{
			\begin{pmatrix}
			1 & 1 \\
			2 & 3 
			\end{pmatrix},
			\begin{pmatrix}
			1 & 3 \\
			2 & 3 
			\end{pmatrix},
			\begin{pmatrix}
			3 & 1 \\
			0 & 3 
			\end{pmatrix},
			\begin{pmatrix}
			3 & 3 \\
			0 & 3 
			\end{pmatrix},
			\right\}	
			\right) $$
		\end{enumerate}
		\item Si $b \equiv c \equiv 1 \mod 2$:
		\begin{align*}
		\phi(\mathcal{M}) = 
		\begin{pmatrix}
		x & 1 \\
		1 & t 
		\end{pmatrix}  & \implies
		\phi(\overline{\mathcal{M}}) = 
		\phi(\mathcal{M}\mathbb{S}) = \left\{
		\begin{pmatrix}
		x & 0 \\
		1 & t 
		\end{pmatrix},
		\begin{pmatrix}
		x & 0 \\
		1 & t+2 
		\end{pmatrix},
		\begin{pmatrix}
		x+2 & 0 \\
		1 & t 
		\end{pmatrix}, 
		\begin{pmatrix}
		x+2 & 0 \\
		1 & t+2 
		\end{pmatrix}
		\right\} \\
		& \implies
		\varphi(\overline{\mathcal{M}}) =
		\left\{
		\begin{pmatrix}
		m & y \\
		z & n
		\end{pmatrix} \suchthat 	
		\begin{aligned}
		m &\in \{x, x+2\} \\
		n &\in \{t, t+2\} \\
		y, z &\in \{1, 3\} 
		\end{aligned}					
		\right\}
		\end{align*}
		Ahora, tenemos que 
		$
		\overline{\mathcal{M}} =
		\varphi^{-1}\left( 
		\left\{
		\begin{pmatrix}
		m & y \\
		z & n
		\end{pmatrix}
		\suchthat 
		\begin{aligned}
		m &\in \{x, x+2\} \\
		n &\in \{t, t+2\} \\
		y, z &\in \{1, 3\} 
		\end{aligned}		
		\right\}	
		\right)$ 		
		y que como 
		$ \begin{pmatrix}
		m & y \\
		z & n
		\end{pmatrix} 
		\in SL_2(\mathbb{Z}/ 4\mathbb{Z}) $, sabemos que $(x, t) = (1,1), (1,3), (3,1) $ o $ (3,3)$.
	\end{enumerate}
	
\end{document}