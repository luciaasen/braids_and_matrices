
\documentclass{article}


\usepackage{lipsum}
\usepackage{lmodern}
\usepackage[T1]{fontenc}
\usepackage[utf8]{inputenc}
\usepackage{mathtools}
\usepackage{amssymb}
\usepackage{amsmath}
\usepackage{graphicx}
\usepackage{sectsty} % Para cambiar el tamanio de los titulos
\usepackage[lmargin=1in, rmargin=0.75in]{geometry}
\usepackage{xcolor}
%\usepackage{hyperref}
\usepackage[colorlinks = true,
linkcolor = blue,
urlcolor  = blue,
citecolor = blue,
anchorcolor = blue]{hyperref}

\sectionfont{\LARGE}
\newcommand\tab[1][0.6cm]{\hspace*{#1}}
\newcommand\nl{\newline\tab}
\renewcommand\labelenumi{\textbf{\arabic{enumi}.}}
\renewcommand\labelenumii{\textbf{\theenumi.\arabic{enumii}.}}
\renewcommand\thesection{EJERCICIO \arabic{section}} %Para que las secciones no se nombren como chapter.section (0.1, 0.2 etc)
%\renewcommand\thesubsection{}
\renewcommand\labelitemi{$\bullet$ }

\setcounter{secnumdepth}{3}

\linespread{1.2}

\title{Mis rayadas con los generadores}
\date{}
\begin{document}
	\maketitle
	%\begin{center}
	Tenemos unos hermosos potenciales generadores de $ SL(2, \mathbb{Z}) $, que son : 
	$$
	x = 
	\begin{pmatrix}
	1 & 1 \\
	0 & 1 
	\end{pmatrix},	
	y = 
	\begin{pmatrix}
	1 & 0 \\
	1 & 1 
	\end{pmatrix}
	$$
	\nl
	Si asumimos que $x, y$ generan todas las $M \in SL(2, \mathbb{Z}) $ con  $ \| M \|_\infty \leq 2 $, ¿podemos demostrar que $\langle x, y \rangle =  SL(2, \mathbb{Z})$? \nl
	%\paragraph{} 
	Veamos que sí, y chachi pistachi. 
	Sea $$
	A = 
	\begin{pmatrix}
	a_{11} & a_{12} \\
	a_{21} & a_{22} 
	\end{pmatrix} \in SL(2, \mathbb{Z}), $$ con  $ \| A \|_\infty = a_{11} > 2
	$ \nl 
	Lo primero para ahorrarnos quebraderos de cabeza es darnos cuenta de que, con las hipótesis anteriores,  $|a_{11}| \neq |a_{ij}|$ para todo $ij \neq 11$, y que no puede ser que $ |a_{22}| > |a_{12}|, |a_{21}|$. También es cierto que $a_{11}a_{22}\geq 0 \iff a_{12}a_{21}\geq 0$.\nl Con esto, la vida es más fácil. \nl
	O yo hago las cosas mal y me creo que es más fácil. Ya veremos. \nl
	Se abren ante nosotros 4 fantásticos caminos en función del orden relativo de los $|a_{ij}|$, cada uno subdividido en otros 4 según el signo de los $a_{ij}$. No son 8 porque hacemos la trampa de que $a_{11} > 0$. \nl
	Además, pa no repetirlo 16 veces, que quede claro que  $|a_{11}| = a, |a_{12}| = b, |a_{21}| = c, |a_{22}| = d $	.
	\begin{enumerate}
		\item  
		$|a_{11}| > |a_{12}| \geq |a_{21}| \geq |a_{22}|   \equiv a > b \geq c \geq d$ \nl
		
		\textbf{Obs.} \nl Como $a > b, |a - b| < a$ \nl
						  Como $c \geq d, |c - d| = |d - c|\leq |c| < a$ 
		\begin{enumerate}
			\item $a_{11} > 2, a_{12} \geq 0, a_{21} \geq 0, a_{22} \geq 0$\nl
			$$
			\begin{aligned}[c]
			AR = 
			\begin{pmatrix}
			a & b \\
			c & d 
			\end{pmatrix} 
			\begin{pmatrix}
			1 & 0 \\
			-1 & 1 
			\end{pmatrix} = 
			\begin{pmatrix}
			a-b & b \\
			c-d & d 
			\end{pmatrix}
			\end{aligned}
			\qquad\qquad
			\begin{aligned}[c]
			|a - b| &< a\\
			|c - d| &< a\\
			\end{aligned}
			$$
			\item $a_{11} > 2, a_{12} \geq 0, a_{21} \leq 0, a_{22} \leq 0$ \nl
			$$
			\begin{aligned}[c]
			AR = 
			\begin{pmatrix}
			a & b \\
			-c & -d 
			\end{pmatrix} 
			\begin{pmatrix}
			1 & 0 \\
			-1 & 1 
			\end{pmatrix} = 
			\begin{pmatrix}
			a-b & b \\
			d-c & -d 
			\end{pmatrix}
			\end{aligned}
			\qquad\qquad
			\begin{aligned}[c]
			|a - b| &< a\\
			|d - c| &< a\\
			\end{aligned}
			$$
			\item $a_{11} > 2, a_{12} \leq 0, a_{21} \geq 0, a_{22} \leq 0$ \nl
			$$
			\begin{aligned}[c]
			AR = 
			\begin{pmatrix}
			a & -b \\
			c & -d 
			\end{pmatrix} 
			\begin{pmatrix}
			1 & 0 \\
			1 & 1 
			\end{pmatrix} = 
			\begin{pmatrix}
			a-b & b \\
			c-d & -d 
			\end{pmatrix}
			\end{aligned}
			\qquad\qquad
			\begin{aligned}[c]
			|a - b| &< a\\
			|c - d| &< a\\
			\end{aligned}
			$$
			\item $a_{11} > 2, a_{12} \leq 0, a_{21} \leq 0, a_{22} \geq 0$ \nl
			$$
			\begin{aligned}[c]
			AR = 
			\begin{pmatrix}
			a & -b \\
			-c & d 
			\end{pmatrix} 
			\begin{pmatrix}
			1 & 0 \\
			1 & 1 
			\end{pmatrix} = 
			\begin{pmatrix}
			a-b & b \\
			d-c & d 
			\end{pmatrix}
			\end{aligned}
			\qquad\qquad
			\begin{aligned}[c]
			|a - b| &< a\\
			|d - c| &< a\\
			\end{aligned}
			$$
		\end{enumerate}
		\item  
		$|a_{11}| > |a_{12}| \geq |a_{22}| \geq |a_{21}| \equiv a > b \geq d \geq c $ \nl
	
		\textbf{Obs.} \nl Como $a > b, |a - b| < a$ \nl
						  Como $d \geq c, |c - d| = |d - c| \leq |d| < a$ 
		\begin{enumerate}
		
			 \item $a_{11} > 2, a_{12} \geq 0, a_{21} \geq 0, a_{22} \geq 0$ \nl
			 $$
			 \begin{aligned}[c]
			 AR = 
			 \begin{pmatrix}
			 a & b \\
			 c & d 
			 \end{pmatrix} 
			 \begin{pmatrix}
			 1 & 0 \\
			 -1 & 1 
			 \end{pmatrix} = 
			 \begin{pmatrix}
			 a-b & b \\
			 c-d & d 
			 \end{pmatrix}
			 \end{aligned}
			 \qquad\qquad
			 \begin{aligned}[c]
			 |a - b| &< a\\
			 |c - d| &< a\\
			 \end{aligned}
			 $$
			 \item $a_{11} > 2, a_{12} \geq 0, a_{21} \leq 0, a_{22} \leq 0$ \nl
			 $$
			 \begin{aligned}[c]
			 AR = 
			 \begin{pmatrix}
			 a & b \\
			 -c & -d 
			 \end{pmatrix} 
			 \begin{pmatrix}
			 1 & 0 \\
			 -1 & 1 
			 \end{pmatrix} = 
			 \begin{pmatrix}
			 a-b & b \\
			 d-c & -d 
			 \end{pmatrix}
			 \end{aligned}
			 \qquad\qquad
			 \begin{aligned}[c]
			 |a - b| &< a\\
			 |d - c| &< a\\
			 \end{aligned}
			 $$
			 \item $a_{11} > 2, a_{12} \leq 0, a_{21} \geq 0, a_{22} \leq 0$ \nl
			 $$
			 \begin{aligned}[c]
			 AR = 
			 \begin{pmatrix}
			 a & -b \\
			 c & -d 
			 \end{pmatrix} 
			 \begin{pmatrix}
			 1 & 0 \\
			 1 & 1 
			 \end{pmatrix} = 
			 \begin{pmatrix}
			 a-b & b \\
			 c-d & -d 
			 \end{pmatrix}
			 \end{aligned}
			 \qquad\qquad
			 \begin{aligned}[c]
			 |a - b| &< a\\
			 |c - d| &< a\\
			 \end{aligned}
			 $$
			 \item $a_{11} > 2, a_{12} \leq 0, a_{21} \leq 0, a_{22} \geq 0$ \nl
			 $$
			 \begin{aligned}[c]
			 AR = 
			 \begin{pmatrix}
			 a & -b \\
			 -c & d 
			 \end{pmatrix} 
			 \begin{pmatrix}
			 1 & 0 \\
			 1 & 1 
			 \end{pmatrix} = 
			 \begin{pmatrix}
			 a-b & b \\
			 d-c & d 
			 \end{pmatrix}
			 \end{aligned}
			 \qquad\qquad
			 \begin{aligned}[c]
			 |a - b| &< a\\
			 |d - c| &< a\\
			 \end{aligned}
			 $$
		\end{enumerate}
		\item  
		$|a_{11}| > |a_{21}| \geq |a_{12}| \geq |a_{22}| \equiv a > c \geq b \geq d $ \nl
			
		\textbf{Obs.} \nl Como $a > c, |a - c| < a$ \nl
		Como $b \geq d, |b - d| = |d - b|\leq |b| < a$ 
		\begin{enumerate}
			
			\item $a_{11} > 2, a_{12} \geq 0, a_{21} \geq 0, a_{22} \geq 0$ \nl
			$$
			\begin{aligned}[c]
			RA = 
			\begin{pmatrix}
			1 & -1 \\
			0 & 1 
			\end{pmatrix} 
			\begin{pmatrix}
			a & b \\
			c & d 
			\end{pmatrix} 
			= 
			\begin{pmatrix}
			a-c & b-d \\
			c & d 
			\end{pmatrix}
			\end{aligned}
			\qquad\qquad
			\begin{aligned}[c]
			|a - c| &< a\\
			|b - d| &< a\\
			\end{aligned}
			$$
			\item $a_{11} > 2, a_{12} \geq 0, a_{21} \leq 0, a_{22} \leq 0$ \nl
			$$
			\begin{aligned}[c]
			RA = 
			\begin{pmatrix}
			1 & 1 \\
			0 & 1 
			\end{pmatrix} 
			\begin{pmatrix}
			a & b \\
			-c & -d 
			\end{pmatrix} 
			= 
			\begin{pmatrix}
			a-c & b-d \\
			c & -d 
			\end{pmatrix}
			\end{aligned}
			\qquad\qquad
			\begin{aligned}[c]
			|a - c| &< a\\
			|b - d| &< a\\
			\end{aligned}
			$$
			\item $a_{11} > 2, a_{12} \leq 0, a_{21} \geq 0, a_{22} \leq 0$ \nl
			$$
			\begin{aligned}[c]
			RA = 
			\begin{pmatrix}
			1 & -1 \\
			0 & 1 
			\end{pmatrix} 
			\begin{pmatrix}
			a & -b \\
			c & -d 
			\end{pmatrix} 
			= 
			\begin{pmatrix}
			a-c & d-b \\
			c & -d 
			\end{pmatrix}
			\end{aligned}
			\qquad\qquad
			\begin{aligned}[c]
			|a - c| &< a\\
			|d - b| &< a\\
			\end{aligned}
			$$
			\item $a_{11} > 2, a_{12} \leq 0, a_{21} \leq 0, a_{22} \geq 0$ \nl
			$$
			\begin{aligned}[c]
			RA = 
			\begin{pmatrix}
			1 & 1 \\
			0 & 1 
			\end{pmatrix} 
			\begin{pmatrix}
			a & -b \\
			-c & d 
			\end{pmatrix} 
			= 
			\begin{pmatrix}
			a-c & d-b \\
			-c & d 
			\end{pmatrix}
			\end{aligned}
			\qquad\qquad
			\begin{aligned}[c]
			|a - c| &< a\\
			|d - b| &< a\\
			\end{aligned}
			$$
		\end{enumerate}
		\item  
		$|a_{11}| > |a_{21}| \geq |a_{22}| \geq |a_{12}|  \equiv a > c \geq d \geq b $ \nl
		
		\textbf{Obs.} \nl Como $a > c, |a - c| < a$ \nl
		Como $d \geq b, |b - d| = |d - b|\leq |b| < a$ 
		\begin{enumerate}
			
			\item $a_{11} > 2, a_{12} \geq 0, a_{21} \geq 0, a_{22} \geq 0$ \nl
			$$
			\begin{aligned}[c]
			RA = 
			\begin{pmatrix}
			1 & -1 \\
			0 & 1 
			\end{pmatrix} 
			\begin{pmatrix}
			a & b \\
			c & d 
			\end{pmatrix} 
			= 
			\begin{pmatrix}
			a-c & b-d \\
			c & d 
			\end{pmatrix}
			\end{aligned}
			\qquad\qquad
			\begin{aligned}[c]
			|a - c| &< a\\
			|b - d| &< a\\
			\end{aligned}
			$$
			\item $a_{11} > 2, a_{12} \geq 0, a_{21} \leq 0, a_{22} \leq 0$ \nl
			$$
			\begin{aligned}[c]
			RA = 
			\begin{pmatrix}
			1 & 1 \\
			0 & 1 
			\end{pmatrix} 
			\begin{pmatrix}
			a & b \\
			-c & -d 
			\end{pmatrix} 
			= 
			\begin{pmatrix}
			a-c & b-d \\
			c & -d 
			\end{pmatrix}
			\end{aligned}
			\qquad\qquad
			\begin{aligned}[c]
			|a - c| &< a\\
			|b - d| &< a\\
			\end{aligned}
			$$
			\item $a_{11} > 2, a_{12} \leq 0, a_{21} \geq 0, a_{22} \leq 0$ \nl
			$$
			\begin{aligned}[c]
			RA = 
			\begin{pmatrix}
			1 & -1 \\
			0 & 1 
			\end{pmatrix} 
			\begin{pmatrix}
			a & -b \\
			c & -d 
			\end{pmatrix} 
			= 
			\begin{pmatrix}
			a-c & d-b \\
			c & -d 
			\end{pmatrix}
			\end{aligned}
			\qquad\qquad
			\begin{aligned}[c]
			|a - c| &< a\\
			|d - b| &< a\\
			\end{aligned}
			$$
			\item $a_{11} > 2, a_{12} \leq 0, a_{21} \leq 0, a_{22} \geq 0$ \nl
			$$
			\begin{aligned}[c]
			RA = 
			\begin{pmatrix}
			1 & 1 \\
			0 & 1 
			\end{pmatrix} 
			\begin{pmatrix}
			a & -b \\
			-c & d 
			\end{pmatrix} 
			= 
			\begin{pmatrix}
			a-c & d-b \\
			-c & d 
			\end{pmatrix}
			\end{aligned}
			\qquad\qquad
			\begin{aligned}[c]
			|a - c| &< a\\
			|d - b| &< a\\
			\end{aligned}
			$$
		\end{enumerate}
	\end{enumerate}
	\tab En todo caso, la matriz $R$ tiene $\|R\|_\infty = \|R^{-1}\|_\infty= 1$ \nl Así que podemos expresar $A$ como el producto de 2 matrices de norma $< a_{11}$, y con esto y un bizcocho, hasta mañana a las 8. \nl
	Ah, bueno.
	Si $\|A_2\|_\infty = a_{22}$, entonces en $A_2$ es la inversa de alguna de las matrices $A$ de arriba, y listo. Si $\|A_3\|_\infty = a_{12}$, entonces $A_3$ es el producto de alguna de las matrices $A$ por $\begin{pmatrix}
	0 & -1 \\
	1 & 0 
	\end{pmatrix} $,
	y también está listo. Y si $\|A_4\|_\infty = a_{21}$, entonces $A_4$ es el la inversa del producto de alguna de las matrices $A$ por $\begin{pmatrix}
	0 & 1 \\
	-1 & 0 
	\end{pmatrix} $ \nl
	Y si $\|A_{caca}\|_\infty = |a_{ij}|$, con $a_{ij} < 0$, pues mira, chico. Me estoy hartando. Pero $A_{caca}$ es el producto de alguna de las $A_i$ por  $\begin{pmatrix}
	-1 & 0 \\
	0 & -1 
	\end{pmatrix} $. Que aunque a mí me costó mucho, el ordenador me dijo en un plis plas que sí estaba generada por $x$ e $y$.\nl \nl
	Menuda pesadilla. Que las matrices de norma 1 y 2 están generadas por $x$ e $y$, te lo vas a creer porque sí. \nl PORQUE YO PASO DE LA LIFE :D
	
	%\end{center}
\end{document}