
\documentclass{article}


\usepackage{lipsum}
\usepackage{lmodern}
\usepackage[T1]{fontenc}
\usepackage[utf8]{inputenc}
\usepackage{mathtools}
\usepackage{amssymb}
\usepackage{amsmath}
\usepackage{graphicx}
\usepackage{sectsty} % Para cambiar el tamanio de los titulos
\usepackage[lmargin=1in, rmargin=0.75in]{geometry}
\usepackage{xcolor}
%\usepackage{hyperref}
\usepackage[colorlinks = true,
linkcolor = blue,
urlcolor  = blue,
citecolor = blue,
anchorcolor = blue]{hyperref}

\sectionfont{\LARGE}
\newcommand\tab[1][0.6cm]{\hspace*{#1}}
\newcommand\nl{\newline\tab}
\renewcommand\labelenumi{\textbf{\arabic{enumi}.}}
\renewcommand\labelenumii{\textbf{\theenumi.\arabic{enumii}.}}
\renewcommand\thesection{EJERCICIO \arabic{section}} %Para que las secciones no se nombren como chapter.section (0.1, 0.2 etc)
%\renewcommand\thesubsection{}
\renewcommand\labelitemi{$\bullet$ }

\setcounter{secnumdepth}{3}

\linespread{1.2}

\title{Mis rayadas con los generadores}

\begin{document}
	\maketitle
	%\begin{center}
	Tenemos unos hermosos potenciales generadores de $ SL(2, \mathbb{Z}) $, que son : 
	$$
	x = 
	\begin{pmatrix}
	1 & 1 \\
	0 & 1 
	\end{pmatrix},	
	y = 
	\begin{pmatrix}
	1 & 0 \\
	1 & 1 
	\end{pmatrix}
	$$
	\nl
	Si asumimos que $x, y$ generan todas las $M \in SL(2, \mathbb{Z}) $ con  $ \| M \|_\infty \leq 2 $, ¿podemos demostrar que $\langle x, y \rangle =  SL(2, \mathbb{Z})$? \nl
	%\paragraph{} 
	Veamos que sí, y chachi pistachi. 
	Sea $$
	A = 
	\begin{pmatrix}
	a_{11} & a_{12} \\
	a_{21} & a_{22} 
	\end{pmatrix} \in SL(2, \mathbb{Z}), $$ con  $ \| A \|_\infty = a_{11} > 2
	$ \nl 
	Lo primero para ahorrarnos quebraderos de cabeza es darnos cuenta de que  $|a_{11}| \neq |a_{ij}|$ para todo $ij \neq 11$, y que no puede ser que $ |a_{22}| > |a_{12}|, |a_{21}|$. También es cierto que $a_{11}a_{22}\geq 0 \iff a_{12}a_{21}\leq 0$.\nl Con esto, la vida es más fácil. \nl
	O yo hago las cosas mal y me creo que es más fácil. Ya veremos. \nl
	Con esto, se abren ante nosotros 4 fantásticos casos en función del orden de los $|a_{ij}|$, cada uno subdividido en otros 4 según el signo de los $a_{ij}$. No son 8 porque hacemos la trampa de que $a_{11} > 0$. \nl
	Además, pa no repetirlo 16 veces, que quede claro que  $|a_{11}| = a, |a_{12}| = b, |a_{21}| = c, |a_{22}| = d $	.
	\begin{enumerate}
		\item  
		$|a_{11}| > |a_{12}| \geq |a_{21}| \geq |a_{22}|  $ \nl
		
		\begin{enumerate}
			\item $a_{11} > 2, a_{12} \geq 0, a_{21} \geq 0, a_{22} \geq 0$ \nl
			$$
			\begin{aligned}[c]
			AR = 
			\begin{pmatrix}
			a & b \\
			c & d 
			\end{pmatrix} 
			\begin{pmatrix}
			1 & 0 \\
			-1 & 1 
			\end{pmatrix} = 
			\begin{pmatrix}
			a-b & b \\
			c-d & c 
			\end{pmatrix}
			\end{aligned}
			\qquad\qquad
			\begin{aligned}[c]
			|a - b| &< a\\
			|c - d| &< |c| < a\\
			\end{aligned}
			$$
		\end{enumerate}
		
	\end{enumerate}
	
	
	
	%\end{center}
\end{document}