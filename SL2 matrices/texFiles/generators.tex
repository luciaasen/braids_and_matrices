
\documentclass{article}


\usepackage{lipsum}
\usepackage{lmodern}
\usepackage[T1]{fontenc}
\usepackage[utf8]{inputenc}
\usepackage{mathtools}
\usepackage{amssymb}
\usepackage{amsmath}
\usepackage{graphicx}
\usepackage{sectsty} % Para cambiar el tamanio de los titulos
\usepackage[lmargin=1in, rmargin=0.75in]{geometry}
\usepackage{xcolor}
%\usepackage{hyperref}
\usepackage[colorlinks = true,
linkcolor = black,
urlcolor  = blue,
citecolor = blue,
anchorcolor = blue]{hyperref}

\sectionfont{\LARGE}
\newcommand{\rpm}{\raisebox{.2ex}{$\scriptstyle\pm$}} % rpm is a smaller pm
\newcommand\tab[1][0.6cm]{\hspace*{#1}}
\newcommand\nl{\newline\tab}
\renewcommand\labelenumi{\textbf{\arabic{enumi}.}}
\renewcommand\labelenumii{\textbf{\theenumi.\arabic{enumii}.}}

\renewcommand{\theenumii}{\textbf{.\arabic{enumii}.}}
%\renewcommand{\labelenumi}{\alph{enumi}}


\renewcommand\thesubsection{\arabic{subsection}.} 
\renewcommand\thesection{\arabic{section}} %Para que las secciones no se nombren como chapter.section (0.1, 0.2 etc)
\renewcommand\labelitemi{$\bullet$ }

\setcounter{secnumdepth}{3}

\linespread{1.2}

\title{Mis rayadas con los generadores}
\date{}
\begin{document}
	\maketitle
	%\begin{center}
	Tenemos unos hermosos potenciales generadores de $ SL(2, \mathbb{Z}) $, que son : 
	$$
	x = 
	\begin{pmatrix}
	1 & 1 \\
	0 & 1 
	\end{pmatrix},	
	y = 
	\begin{pmatrix}
	1 & 0 \\
	1 & 1 
	\end{pmatrix}
	$$
	
	\subsection{Caso base}
	\tab Como me siento generosa, vamos ver que las matrices de norma 1 realmente están generadas por $x$ e $y$.\nl 
	Asumimos que $\|A\|_\infty = a_{11} = 1$, y los casos $\|A\|_\infty = \pm a_{ij}$ nos vienen de regalo. Y si no te lo crees, pues te lees la inducción o te jodes. As you wish, oye. Estamos en un documento libre.
	\begin{enumerate}
		\item
		$
		A = 
		\begin{pmatrix}
		1 & 0 \\
		0 & 1 
		\end{pmatrix} 
		= I
		$
		\item
		$
		A = 
		\begin{pmatrix}
		1 & 1 \\
		0 & 1 
		\end{pmatrix} 
		= x
		$
		\item
		$
		A = 
		\begin{pmatrix}
		1 & 0 \\
		1 & 1 
		\end{pmatrix} 
		= y
		$
		\item
		$
		A = 
		\begin{pmatrix}
		1 & -1 \\
		0 & 1 
		\end{pmatrix} 
		= x^{-1}
		$
		\item
		$
		A = 
		\begin{pmatrix}
		1 & -1 \\
		1 & 0 
		\end{pmatrix} 
		= yx^{-1}
		$			
		\item
		$
		A = 
		\begin{pmatrix}
		1 & 0 \\
		-1 & 1 
		\end{pmatrix} 
		= y^{-1}
		$
		\item
		$
		A = 
		\begin{pmatrix}
		1 & 1 \\
		-1 & 0 			
		\end{pmatrix} 
		= y^{-1}x			
		$					
		
	\end{enumerate}
	
	Por último, para poder generalizar estos casos a  $\|A\|_\infty = \pm a_{ij}$ como se hará en la inducción, dejo caer que
	$\begin{pmatrix}
	0 & -1 \\
	1 & 0 
	\end{pmatrix} = yx^{-1}y $, que $\begin{pmatrix}
	0 & 1 \\
	-1 & 0 
	\end{pmatrix} = \left(yx^{-1}y\right)^{-1}$ y que 
	$\begin{pmatrix}
	-1 & 0 \\
	0 & -1 
	\end{pmatrix} = \left(y^{-1}x^2\right)^2$.\nl  Vaya, que las tres están en $ \langle x, y \rangle$. Así que no me vengas luego con tonterías.\nl\nl
	
	
	
	\subsection{Antiquebraderos de cabeza}
	\tab Es conveniente, por el bien de tu mollera, que antes de seguir observes que las siguientes afirmaciones son ciertas si $\|A\|_\infty = a_{11} > 1$:
	
	\begin{equation}\label{a}
	|a_{11}| > |a_{ij}| \text{ para } ij \neq 11 
	\end{equation}
	%\tab 
	\nl 
	\textit{¿Por qué?}\nl 
	Si $\quad a_{12} = a_{11}$ (el caso $a_{21} = a_{11}$ es idéntico),\nl 
	\tab$1 = |A| = |a_{11}(a_{22}-a_{21})| \geq 2|a_{22}-a_{21}|$, pero entonces $\frac{1}{2} \geq |a_{22}-a_{21}|$ y caca porque $a_{ij} \in\mathbb{Z} $.\nl
	Si $\quad a_{22} = a_{11} (\implies a_{12}, a_{21} < a_{11},\quad$ porque si no $\quad|A| \geq a_{11} > 1)$,\nl 
	\tab $1 = |A| = |a_{11}^2-a_{12}a_{21}| \geq |a_{11}^2-(a_{11}-1)^2| = |2a_{11} - 1 | > 1$, caca.\nl\nl
	\begin{equation}\label{b}
	|a_{22}| \leq |a_{12}| \text{ ó } |a_{22}| \leq|a_{21}|
	\end{equation}
	%\tab 
	\nl 
	\textit{¿Por qué?}\nl 
	 Si $\quad|a_{22}| > |a_{12}|, |a_{21}|\quad$  y utilizando que por (\ref{a}), $\quad |a_{11}| > |a_{22}|$, \nl 
	\tab $1 = |A| = |a_{11}a_{22}-a_{12}a_{21})| > |a_{22}(a_{11}-a_{22})| \geq |1(2 - 1)| > 1 $, caca.\nl\nl
	\begin{equation}\label{c}
	a_{11}a_{22}\geq 0 \iff a_{12}a_{21}\geq 0\text{, para } a_{22} \neq 0
	\end{equation}
	\nl 
	\textit{¿Por qué?}\nl 
	%\tab
	Si $\quad a_{11}a_{22} > 0, (\implies a_{11}a_{22} > 1,\quad$ porque $\quad a_{11} > 1, a_{22} \neq 0)$\nl\tab 
	$1 = |A| = a_{11}a_{22} - a_{12}a_{21} > 1 - a_{12}a_{21} \quad\implies\quad a_{12}a_{21} > 0$\nl
	Si $\quad a_{11}a_{22} < 0 (\implies a_{11}a_{22} < - 1,\quad$ porque $\quad a_{11} > 1, a_{22} 	\neq 0)$\nl\tab 
	$1 = |A| = a_{11}a_{22} - a_{12}a_{21} < - 1 - a_{12}a_{21} \quad\implies\quad a_{12}a_{21} < -2 < 0$\nl\nl 
	Hating \LaTeX\space  learning curve.\nl	Más te vale apreciar que haya escrito esas demostraciones, aunque estén mal. Porque estoy llenando el documento de chapuzas para ello. -.-"\nl 	
	En fin. Con esto, la vida es más fácil. Y con lo que dijo Yago de las traspuestas, pues todavía más.\nl
	\subsection{Inducción }
	\tab Si asumimos que $x, y$ generan todas las $M \in SL(2, \mathbb{Z}) $ con  $ \| M \|_\infty < a_{11} $, ¿podemos demostrar que $\langle x, y \rangle =  SL(2, \mathbb{Z})$? 
	%\paragraph{} 
	Veamos que sí, y chachi pistachi. 
	Sea $$
	A = 
	\begin{pmatrix}
	a_{11} & a_{12} \\
	a_{21} & a_{22} 
	\end{pmatrix} \in SL(2, \mathbb{Z}), $$ con  $ \| A \|_\infty = a_{11} > 1
	$ \nl 
	Estudiamos $A$ en función de $a_{22}$, del orden relativo de los $|a_{ij}|$ y de sus signos.\nl
	Sean $|a_{11}| = a, |a_{12}| = b, |a_{21}| = c, |a_{22}| = d $	.
	\begin{enumerate}
		\item $ a_{22} \neq 0$
			\begin{enumerate}
				
			\item \label{1}  
			$|a_{11}| > |a_{12}| \geq |a_{21}|, |a_{22}|   \equiv a > b \geq c, d$ \nl
			
			\textbf{Obs.} \nl Como $a > b, |a - b| < a$ \nl
							  Si $c \geq d, |c - d| = |d - c|\leq |c| < a$
							  \nl 
			Si  $d \geq c, |c - d| = |d - c|\leq |d| < a$
			\begin{enumerate}
				\item  $a_{11} > 1, a_{12} \geq 0, a_{21} \geq 0, a_{22} > 0 $\nl
				$$
				%\begin{aligned}[b]
				AR = 
				\begin{pmatrix}
				a & b \\
				c & d 
				\end{pmatrix} 
				\begin{pmatrix}
				1 & 0 \\
				-1 & 1 
				\end{pmatrix} = 
				\begin{pmatrix}
				a-b & b \\
				c-d & d 
				\end{pmatrix}
				%\end{aligned}
				\qquad\qquad
				\begin{aligned}[c]
				|a - b| &< a\\
				|c - d| &< a\\
				\end{aligned}
				$$
				\item $a_{11} > 1, a_{12} \geq 0, a_{21} \leq 0, a_{22} < 0$ \nl
				$$
				%\begin{aligned}[c]
				AR = 
				\begin{pmatrix}
				a & b \\
				-c & -d 
				\end{pmatrix} 
				\begin{pmatrix}
				1 & 0 \\
				-1 & 1 
				\end{pmatrix} = 
				\begin{pmatrix}
				a-b & b \\
				d-c & -d 
				\end{pmatrix}
				%\end{aligned}
				\qquad\qquad
				\begin{aligned}[c]
				|a - b| &< a\\
				|d - c| &< a\\
				\end{aligned}			
				$$
				\item $a_{11} > 1, a_{12} \leq 0, a_{21} \geq 0, a_{22} < 0$ \nl
				$$
				%\begin{aligned}[c]
				AR = 
				\begin{pmatrix}
				a & -b \\
				c & -d 
				\end{pmatrix} 
				\begin{pmatrix}
				1 & 0 \\
				1 & 1 
				\end{pmatrix} = 
				\begin{pmatrix}
				a-b & b \\
				c-d & -d 
				\end{pmatrix}
				%\end{aligned}
				\qquad\qquad
				\begin{aligned}[c]
				|a - b| &< a\\
				|c - d| &< a\\
				\end{aligned}
				$$
				\item $a_{11} > 1, a_{12} \leq 0, a_{21} \leq 0, a_{22} > 0$ \nl
				$$
				%\begin{aligned}[c]
				AR = 
				\begin{pmatrix}
				a & -b \\
				-c & d 
				\end{pmatrix} 
				\begin{pmatrix}
				1 & 0 \\
				1 & 1 
				\end{pmatrix} = 
				\begin{pmatrix}
				a-b & b \\
				d-c & d 
				\end{pmatrix}
				%\end{aligned}
				\qquad\qquad
				\begin{aligned}[c]
				|a - b| &< a\\
				|d - c| &< a\\
				\end{aligned}
				$$
			\end{enumerate}
			
			En los cuatro casos, $\| AR \|_\infty < \|A\|_\infty$\nl
			
			
			\item  
			$|a_{11}| > |a_{21}| \geq |a_{12}|, |a_{22}| \equiv a > c \geq b, d $ \nl
			
			
			\textbf{Obs.} \nl  Si $|a_{21}| \geq |a_{12}|$, 
			$
			A^T = 
			\begin{pmatrix}
			a_{11} & a_{21} \\
			a_{12} & a_{22} 
			\end{pmatrix} $ 
			está en las hipótesis de \textbf{\ref{1}}\nl		
			Por tanto,
			$
			\| A^TR \|_\infty < \|A\|_\infty 
			$
			, y entonces,
			$ \| R^TA \|_\infty = \| \left( A^TR\right) ^T \|_\infty = \| A^TR \|_\infty < \|A\|_\infty 
			$.\nl
			
			
			\end{enumerate}	
		\item $a_{22} = 0$
			\begin{enumerate}
				\item $a_{12} \geq 0$\nl
				$$AR = 
				\begin{pmatrix}
					a & b \\
					\pm c & 0 
				\end{pmatrix} 
				\begin{pmatrix}
					1 & 0 \\
					-1 & 1 
				\end{pmatrix} = 
				\begin{pmatrix}
					a-b & b \\
					\pm c & 0 
				\end{pmatrix}
				%\end{aligned}
				\qquad\qquad
				\begin{aligned}[c]
					|a - b| &< a\\
				\end{aligned}
				$$
				\item $a_{12} \leq 0$\nl
				$$AR = 
				\begin{pmatrix}
				a & -b \\
				\pm c & 0 
				\end{pmatrix} 
				\begin{pmatrix}
				1 & 0 \\
				1 & 1 
				\end{pmatrix} = 
				\begin{pmatrix}
				a-b & b \\
				\pm c & 0 
				\end{pmatrix}
				%\end{aligned}
				\qquad\qquad
				\begin{aligned}[c]
				|a - b| &< a\\
				\end{aligned}
				$$\nl
			\end{enumerate}
	\end{enumerate}			

	Además, $\|R\|_\infty = \|R^T\|_\infty = \|R^{-1}\|_\infty = \|\left( R^T\right) ^{-1}\|_\infty=1$, así que podemos expresar $A$ como el producto de 2 matrices de norma $< a_{11}$.\nl\nl
	Y con esto y un bizcocho, hasta mañana a las 8 gracias a la magnífica inducción. \nl
	Vale, no. ¿Y si $a_{11} < 0$? ¿Y si $\|A\|_\infty \neq |a_{11}|$? ¿Eh? ¿EH? ¿Qué patrañas me está contando? \nl
	Mantenga usted la calma, viejo loco. Ahora vamos.\nl\nl
	Si $\|A_2\|_\infty = a_{22}$, entonces $A_2$ es la inversa de alguna de las matrices $A$ de arriba, y listo.\nl
	Si $\|A_3\|_\infty = a_{12}$, entonces $A_3$ es el producto de alguna de las matrices $A$ por
	$\begin{pmatrix}
	0 & -1 \\
	1 & 0 
	\end{pmatrix} $,
	y también está listo.\nl 
	Y si $\|A_4\|_\infty = a_{21}$, entonces $A_4$ es el producto de
	$\begin{pmatrix}
	0 & 1 \\
	-1 & 0 
	\end{pmatrix} $ por alguna de las matrices $A$.\nl
	Por último, si $\|A_{caca}\|_\infty = |a_{ij}|$, con $a_{ij} < 0$... Pues mira, chico. Tanta exhaustividad acaba con mi paciencia. $A_{caca}$ es el producto de alguna de las $A_i$ por  $\begin{pmatrix}
	-1 & 0 \\
	0 & -1 
	\end{pmatrix} $. Que aunque a mí me costó mucho, el ordenador me dijo en un plis plas que sí estaba generada por $x$ e $y$. Lo que hay que aguantar, ¿eh? 	\nl 		
	Jo. Estoy muerta :c... A momi.
	
\end{document}