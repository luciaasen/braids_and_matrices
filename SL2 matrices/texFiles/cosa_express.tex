
\documentclass{article}

\usepackage{tikzsymbols}
\usepackage{lipsum}
\usepackage{lmodern}
\usepackage[T1]{fontenc}
\usepackage[utf8]{inputenc}
\usepackage{mathtools}
\usepackage{amssymb}
\usepackage{amsmath}
\usepackage{graphicx}
\usepackage{sectsty} % Para cambiar el tamanio de los titulos
\usepackage[lmargin=1in, rmargin=0.75in]{geometry}
\usepackage{xcolor}
%\usepackage{hyperref}
\usepackage[colorlinks = true,
linkcolor = black,
urlcolor  = blue,
citecolor = blue,
anchorcolor = blue]{hyperref}

\sectionfont{\LARGE}
\newcommand{\rpm}{\raisebox{.2ex}{$\scriptstyle\pm$}} % rpm is a smaller pm
\newcommand\tab[1][0.6cm]{\hspace*{#1}}
\newcommand\nl{\newline\tab}
\newcommand{\bigslant}[2]{{\raisebox{.2em}{$#1$}\left/\raisebox{-.2em}{$#2$}\right.}}
\newcommand{\suchthat}{\;\ifnum\currentgrouptype=16 \middle\fi|\;}

\renewcommand\labelenumi{\textbf{\arabic{enumi}.}}
\renewcommand\labelenumii{\textbf{\theenumi.\arabic{enumii}.}}

\renewcommand{\theenumii}{\textbf{.\arabic{enumii}.}}
%\renewcommand{\labelenumi}{\alph{enumi}}


\renewcommand\thesubsection{\arabic{subsection}.} 
\renewcommand\thesection{\arabic{section}} %Para que las secciones no se nombren como chapter.section (0.1, 0.2 etc)
\renewcommand\labelitemi{$\bullet$ }

\setcounter{secnumdepth}{3}

\linespread{1.2}

\title{El levante incita al suicidio}
\date{}
\begin{document}
	\maketitle
	A ve.\nl 
	Yo quería ver $ \bigslant{SL_2(\mathbb{Z})}{\mathbb{S}}$, ¿no?  
	
	(Con $\mathbb{S} = 
	\langle
	\begin{pmatrix}
	1 & 2 \\
	0 & 1 
	\end{pmatrix},
	\begin{pmatrix}
	1 & 0 \\
	2 & 1 
	\end{pmatrix}
	\rangle $)
	\nl
	El caso es que si $\varphi$ es la función ésta que hacía el módulo 4, la clase en  
	$ \bigslant{SL_2(\mathbb{Z})}{\mathbb{S}} $ a la que pertenecía una matriz $\mathcal{M} \in SL_2(\mathbb{Z})$ resultaba ser
	$$ 
	\overline{\mathcal{M}} = \varphi^{-1}\left( \varphi(\mathcal{M})\varphi(\mathbb{S})\right)
	$$
	\tab A lo mejor eso es mentira, pero estoy bastante convencida de que es verdad, y como $\varphi(\mathbb{S})$ son sólo 4 elementos, pues era muy cómodo mirar todos los posibles $ \varphi(\mathcal{M})\varphi(\mathbb{S}) $, que es estudiar $ \bigslant{ \varphi(SL_2(\mathbb{Z}))}{ \varphi(\mathbb{S}) } =  \bigslant{SL_2(\mathbb{Z}/4\mathbb{Z})}{ \varphi(\mathbb{S}) }$ que, como has dicho, se hace muy bien por fuerza bruta.\nl
	Y con esas, sage me divide $ SL_2(\mathbb{Z}/4\mathbb{Z}) $, que tiene 48 elementos, en 12 clases de equivalencia de 4 elementos cada una. Y como lo hace sage, pues supongo que está bien. \nl 
	Así que hurgo felizmente entre mis apuntes en busca de la clasificación de grupos de orden 12 (nunca pensé que la usaría pa na).\nl 
	Y, por ejemplo, para mirar si algún $aba = b$, me ha hecho falta multiplicar una clase por otra.\nl
	Y mi desesperación ha sido que, con las clases que sage me había calculado, podía encontrar clases $\overline{\mathcal{M}_1} = \{ \mathcal{M}_{11}, \mathcal{M}_{12}, \mathcal{M}_{13}, \mathcal{M}_{14} \}$  y $\overline{\mathcal{M}_2} = \{ \mathcal{M}_{21}, \mathcal{M}_{22}, \mathcal{M}_{23}, \mathcal{M}_{24} \}$ tales que $\overline{\mathcal{M}_1}\overline{\mathcal{M}_2}$ (hacer las 16 posibles multiplicaciones $\mathcal{M}_{1i}\mathcal{M}_{2j}$), que debería resultar en una tercera clase de equivalencia $\overline{\mathcal{M}_3} = \{ \mathcal{M}_{31}, \mathcal{M}_{32}, \mathcal{M}_{33}, \mathcal{M}_{34} \}$, devolviera en realidad 8 matrices distintas de $SL_2(\mathbb{Z}/4\mathbb{Z})$.\nl Y como las clases de equivalencia tienen 4 elementos, pues eso está mal. Y me desconcierta un poco.\nl 
	Porque el código es tan sencillo que no entiendo dónde he podido pifiarla, y no sé si hay alguna cosa que yo esté pasando por alto by the face. 
	
	Así que:
	\begin{itemize}
		\item O soy subnormal perdida y no sé escribir un bucle que multiplique matrices,
		\item O hay algo que entiendo rematadamente mal y no me doy cuenta,
		\item O lo único de toda esta historia que he asumido por la cara es falso y $\mathbb{S}$ no es normal en $SL_2(\mathbb{Z})$, y entonces el cociente no tiene la estructura de grupo que yo andaba buscando.
	\end{itemize}
	\begin{center}
		\Laughey[1.4]
	\end{center}
\end{document}